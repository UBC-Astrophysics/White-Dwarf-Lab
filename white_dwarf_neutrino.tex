\documentclass{article}
\usepackage[pdftex]{graphicx}
\graphicspath{{../images/}}
\usepackage{boxedminipage}
\usepackage[hints]{ms}
%\usepackage{ms}
%\usepackage{tikz}
%\usepackage[utf8]{inputenc}
\begin{document}
\begin{titlebox}{White Dwarfs and Neutrinos}
Ilaria Caiazzo, Jeremy Heyl \\
TAs: Xianfei Zhang, Sarafina Nance, Ilka Petermann
\end{titlebox}

\section{Introduction}

The cooling of young white dwarfs is dominated by neutrinos, so young white dwarfs are a great probe of weak interactions.  In this lab, you are going to extend MESA to vary the neutrino emission rates using \texttt{run\_star\_extras} and a parameter that you can set in the \texttt{inlist}.

\section{\texttt{run\_star\_extras}}

\textbf{Task:}
\begin{enumerate}
 \setlength\itemsep{0em}
 \item 
Make a copy of your working directory from the previous lab to build your neutrino code.
\item 
Find the correct routines to change in \texttt{run\_star\_extras.f}.
\item 
Using a parameter that you can set in the inlist, multiply the neutrino rate and its derivatives by the parameter.
\item
Run the evolution of the best-fitting model from the previous lab with various neutrino rates, 0.1, 0.3, 3 and 10 times the standard rate.
\item Use \texttt{paintisochone.py} to calculate the absolute magnitudes of your model white dwarfs.
\end{enumerate}

\textbf{Bonus Task:}
MESA keeps track of all of the various neutrino rates.  Multiply just the plasma neutrinos and add the extra neutrino production to the totals.

\hint{\textbf{Hint:}

Which part of \texttt{run\_star\_extras.f} will have the neutrino routines?
}

\hint{\textbf{Hint:}

You may have to change the number of retries, backups and varcontrol to get the new models to work.
}

\section{The Evolution}

\textbf{Task:}\vspace{-1em}
\begin{enumerate}
 \setlength\itemsep{0em}
\item Let's first look at the track of luminosity against effective temperature.  Does varying the neutrinos have an effect?
\item Let's look at luminosity against core temperature.  Does varying the neutrinos have an effect?
\item Let's look at luminosity against time. 
 Does varying the neutrinos have an effect?
 \item 
Which set of observations from the previous lab could probe the effect of neutrinos?  Make the plots using the young and old white dwarf data files.
\end{enumerate}

\hint{\textbf{Hint:}

The cumulative luminosity function measures the cooling evolution of the white dwarfs.  The young ones are bright in the ultraviolet.
}

\end{document}
